\documentclass{article}
\usepackage{z-eves}

This is Spivey's Birthday Book example, suitable for processing with
Z/EVES Version 2, and illustrating the analysis of a complete specification.
It can be imported into the GUI, or ``read'' in the command line interface.

The Birthday Book is a system for recording people's birthdays.
There are operations for adding or removing people, 
or finding the names of people with a given birthday.

The sets $NAME$ of people's names, and $DATE$, of dates, are taken as
given; their structure is of no concern for this detail of specification.

\begin{zed}
[NAME, DATE]
\end{zed}

The Birthday Book uses a partial function $birthday$ to record the
birthdays of known people.  The set $known$ constains the names of
the people whose birthdays are recorded.

\begin{schema}{BirthdayBook}
  known: \power NAME\\
  birthday: NAME \pfun DATE
\where
  known = \dom birthday
\end{schema}

The following property follows immediately from the definition of
$BirthdayBook$.  It is stated as a forward rule to allow several of the
proofs that follow to succeed without the need to expand the BirthdayBook
schema.

\begin{theorem}{frule BirthdayBookPredicate}
  BirthdayBook \implies known = \dom birthday
\end{theorem}
\begin{zproof}
reduce;
\end{zproof}

Initially, there are no recorded birthdays.

\begin{schema}{InitBirthdayBook}
  BirthdayBook
\where
  birthday = \emptyset \\
  known = \emptyset
\end{schema}

The initial state is satisfiable:

\begin{theorem}{InitIsOK}
  \exists BirthdayBook @ InitBirthdayBook
\end{theorem}

\begin{zproof}
prove by reduce;
\end{zproof}

The $AddBirthday$ operation registers a new birthday, given a
name and a date.

\begin{schema}{AddBirthday}
  \Delta BirthdayBook \\
  name?: NAME \\
  date?: DATE
\where
  name? \notin known
\also
  birthday' = birthday \cup \{ name? \mapsto date? \}
\end{schema}

\begin{theorem}{AddBirthdayIsHonest}
\forall BirthdayBook; name?: NAME; date?: DATE @
     name? \notin known \implies \pre AddBirthday
\end{theorem}

\begin{zproof}
prove by reduce;
\end{zproof}

Although the specification of $AddBirthday$ does not mention the final
value of $known$, it is obviously the initial value with the addition of
the new name:

\begin{theorem}{knownAddBirthday}
  AddBirthday \implies known' = known \cup \{ name? \}
\end{theorem}

\begin{zproof}
prove by reduce;
\end{zproof}

The $Find$ operation returns the birthday of a particular user.
on that date.

\begin{schema}{FindBirthday}
  \Xi BirthdayBook  \\
  name?: NAME\\
  date!: DATE
\where
  name? \in known
\also
  date! = birthday (name?)
\end{schema}

The domain condition can be proved with the simplify command.
\begin{zproof}[FindBirthday\$domainCheck]
simplify;
\end{zproof}

% simple theorem: find after add gives the expected date:

\begin{zed}
  AddThenFind \defs AddBirthday \semi FindBirthday
\end{zed}

\begin{theorem}{AddFind}
  AddThenFind \implies date! = date?
\end{theorem}

\begin{zproof}
prove by reduce;
\end{zproof}

The $Remind$ operation returns the set of names of people with
birthdays on a given date.

\begin{schema}{Remind1}
  \Xi BirthdayBook \\
  today?: DATE \\
  cards!: \power NAME
\where
  cards! = \{ n: NAME | birthday(n) = today? \}
\end{schema}
\begin{zproof}[Remind1\$domainCheck]
prove;
\end{zproof}

Rewriting the domain condition leads to an obviously unprovable goal.
Oops.  In $\{ n: NAME | birthday(n) = today? \}$ we are considering
$birthday(n)$ for every $n \in NAME$.  But $birthday$ is partial.

Here is another try:

\begin{schema}{Remind}
  \Xi BirthdayBook \\
  today?: DATE \\
  cards!: \power NAME
\where
  cards! = \{ n: known | birthday(n) = today? \}
\end{schema}

Now, the prove command finishes the proof of the domain check.
\begin{zproof}[Remind\$domainCheck]
prove;
\end{zproof}

Schema calculus is supported.  Here is a "total" AddBirthday operation

\begin{zed}
  REPORT::= ok | already\_known | not\_known
\end{zed}

\begin{schema}{Success}
  result!: REPORT
\where
  result! = ok
\end{schema}

\begin{schema}{AlreadyKnown}
  \Xi BirthdayBook \\
  name?: NAME \\
  result!: REPORT
\where
   name? \in known \\
   result! = already\_known
\end{schema}

The "total" AddBirthday operation:

\begin{zed}
  RAddBirthday \defs (AddBirthday \land Success) \lor AlreadyKnown
\end{zed}

\begin{theorem}{RAddBirthdayIsTotal}
  \forall BirthdayBook; name?: NAME; date?: DATE
    @ \pre RAddBirthday
\end{theorem}
\begin{zproof}
prove by reduce;
split name? \in known;
prove;
\end{zproof}

\end{document}

